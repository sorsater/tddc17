\documentclass[12pt,a4paper]{article}
\usepackage[utf8]{inputenc}
\usepackage{fancyhdr}
\usepackage{datetime2}
\usepackage{parskip}

\pagestyle{fancy}
\fancyhf{}

\lhead{eriro331, micso554}
\rhead{\today} % yyyy-mm-dd
\setlength{\headheight}{15pt}

\cfoot{\thepage}

\begin{document}

\begin{center}
    \Huge
    \textbf{TDDC17 - AI}

    \vspace{0.3cm}
    \Large
    Erik Rönmark
    Michael Sörsäter
    
    \vspace{0.7cm}
    \textbf{Lab1 report}
\end{center}

Our agent works the same in both Task1 and Task2.
The main concept is to search for an unknown position with breadth first search, go there and repeat until done.

The class MyAgentState has mainly the following fields added to it:
\begin{itemize}
	\item \textit{nodeWorld}, a 30x30 array that contains our own class Node (described below)
	\item \textit{qNodes}, a queue used for storing nodes during the breadth first search
	\item \textit{mapDone}, a Boolean used for knowing when map is done
	\item some parameters to keep track of if we find a wall, to optimize the search
\end{itemize}

The class Node contains of the following parameters:
\begin{itemize}
	\item \textit{distance} - the cost for the Node
	\item \textit{x/y} - the x/y-position for the Node
	\item \textit{parent\_x/\_y} - the x/y-position for the parent Node
\end{itemize}

The main decision loop consists of the following steps:
\begin{itemize}
	\item If we don't have a goal position, use BFS - Breadth First Search to find an close \textit{unknown} position and set that to the goal position. Go towards that position
	\item If there is dirt on the way, remove it
	\item If we bump, recalculate to new position with BFS to compensate for walls
	\item If there is no more \textit{unknown} positions, go back home and power off
\end{itemize}

We chose BFS because it is easy to implement the algorithm and it is good enough for finding a close position. Oftentimes the desired position is just a few steps away.

First we thought that sweeping through the room was a good idea. Then we thought about obstacles and realized that it needed to be a lot of cases to go round an obstacle, we changed our minds and instead implemented BFS.

Some optimizations that are implemented is to prefer to go in the direction ahead of the robot. That is done because rotation is so expensive and we save a lot of actions to do that little trick.

Another is if we bump in to 4 walls either south or east, that will be our new limit for searching. So we will not bump into all the walls to check if there is walls there.

If we would do the lab again, we would begin with the search algorithm to save time.

\end{document}